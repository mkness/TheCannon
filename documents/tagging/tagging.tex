% This file is part of the Apogee project.
% Copyright 2014 the authors.

\documentclass[12pt, preprint]{aastex}

\newcommand{\sectionname}{Section}
\newcommand{\project}[1]{\textsl{#1}}

\begin{document}

\title{Data-driven stellar ages and chemical tags}
\author{
  some combination of DWH, MKN, HWR, and others
}

\begin{abstract}
We perform simple age determinations and chemical tagging of stars
without the use of \emph{any} stellar models or model spectra from
libraries.
\end{abstract}

\section{introduction}

The idea of chemical tagging is that every stellar birth environment
(every molecular cloud) is expected to have slightly different
chemical abundances in detail from every other birth environment, such
that stars with common origins but different present-day phase-space
locations can be identified by their surface chemical abundances.
This potentially revolutionary scientific program has not yet
delivered results, in part because it seems to require tens or
hundreds of thousands of high resolution, high signal-to-noise stellar
spectra, and in part because it requires stellar models good enough to
make consistent measurements of abundances across a wide range of
spectral types (temperatures and surface gravities).

There are a range of near-future projects obtaining the required
spectra, including \project{APOGEE} (cite) and \project{HERMES}
(cite).  There are efforts to analyze and improve stellar models
(cite).  However, it is possible at the present day---with spectra in
hand and without trustworthy models at all---to assess the feasiblity
of chemical tagging projects.  The idea of the present project is the
following: If sufficient tagging information resides within high
resolution and high signal-to-noise spectra, then that information
will be visible in the spectra themselves.  Models might be required
to \emph{interpret} the information, but they are not required to
\emph{identify its existence}.  Since abundance measurements are made
in the spectra, if the information to perform precise chemical tagging
is not present in the spectra, it cannot be present in any abundance
measurements derived therefrom.

In what follows, we use machine-learning methods that have been
designed to find structure in large, high dimensional data sets to
look for structure in stellar spectra.  We perform a set of staged
experiments, starting with simple projects to produce data-driven age
indicators, alpha-abundance indicators, and cluster-membership
indicators.  We finish with an assessment of the feasibility of the
most ambitious chemical tagging programs.

...Related prior work includes PCA, HMF, XD, SVM.  These all build
data-driven models of non-trivial data sets, and all could be used
here.  We are going to try to stick with methods that involve building
probabilistic generative models...

\section{generalities}

Data-driven models can be implemented for problems like this in a
number of ways.  In this \sectionname, we describe three formulations
of the problems---as a set of classifications, as a regression, and as
density estimation in data space---and remark on technical approaches
for each formulation.

\paragraph{classification formulation:}
...Classification description of the problem.  We have stars from two
clusters; we want to find the directions or subspace of the data space
in which the clusters are \emph{predictably} best classified.  This
method is supervised in that it requires cluster identities up-front
(for at least some stars).  The harder classification problem is to
take stars from one cluster and then stars from \emph{all other
  sources}.  In what subspace are the stars from the cluster separated
from everything else?  As for methods: Compare SVM with a (modified,
heteroskedastic) Factor Analyzer; why is some modification of the FA
so much better for this problem?

\paragraph{regression formulation:}
...Regression description of the problem.  We have stars in a range of
ages or alpha-abundances, and we either start with known ages or alpha
enhancements (the supervised version of the problem) or else we start
with an emission-line or absorption-line profile that indicates age or
alpha enhancement (the unsupervised version of the problem).  In the
supervised version, we fit every star as a base spectrum plus age
indicator components.  In the unsupervised version, we fit every star
as a base spectrum plus known profile.  The base spectrum, in each
case, is a data-driven model, or the best you can do with all the data
you have ever seen.  HMF is good for this.  But if you want to
marginalize out stellar base spectrum, then you need to build priors.
Hierarchical is the best way to do that.

\paragraph{density estimation formulation:}
...Density estimation description of the problem.  We have a large
number of stars and we try to describe the distribution in spectral
space with a mixture of lines or hypersurfaces.  Show that the stars
lie on a finite number (hundreds or thousands) of well-separated
hypersurfaces and don't fill the bulk.  That's hard, but might be
possible if you use existing models as a guide or initialization.

\section{data and objectives}

\section{results and discussion}

\acknowledgements It is a pleasure to thank
  Ross Fadely,
  Rob Fergus,
  Dan Foreman-Mackey
for valuable discussions.

\end{document}
