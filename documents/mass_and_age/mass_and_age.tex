\documentclass[12pt, preprint]{aastex}

% words
\newcommand{\project}[1]{\textsl{#1}}
\newcommand{\thecannon}{\project{The~Cannon}} 
\newcommand{\apogee}{\project{APOGEE}}
\newcommand{\corot}{\project{Corot}}
\newcommand{\kepler}{\project{Kepler}}
\newcommand{\gaia}{\project{Gaia}}
\newcommand{\most}{\project{MOST}}
\newcommand{\documentname}{\textsl{Article}}

% math
\newcommand{\numax}{\nu_{\max}}
\newcommand{\deltanu}{\Delta\nu}

\begin{document}

\title{Measuring red-giant masses and ages with stellar spectra}
\author{M.~Ness\altaffilmark{1},
David~W.~Hogg\altaffilmark{1,2,3},
H.-W.~Rix\altaffilmark{1},
\textbf{others}}
\altaffiltext{1}{Max-Planck-Institut f\"ur Astronomie, K\"onigstuhl 17, D-69117 Heidelberg, Germany}
\altaffiltext{2}{Center for Cosmology and Particle Physics, Department of Phyics,
             New York University, 4 Washington Pl., room 424, New York, NY, 10003, USA}
\altaffiltext{3}{Center for Data Science, New York University, 726 Broadway, 7th Floor, New York, NY 10003, USA}
% \altaffiltext{4}{NSF Astronomy and Astrophysics Postdoctoral Fellow}
% \altaffiltext{5}{Department of Physics \& Astronomy, Johns Hopkins University, Baltimore, MD, 21218, USA}
\email{ness@mpia.de}

\begin{abstract}%
% Context
With \thecannon, we have demonstrated that it is possible to use a
small training set of stars with noisy spectral data and known
stellar-parameter labels to build a data-driven probabilistic model of
stellar spectra that can be used to infer stellar-parameter labels for
other stars (with differently noisy spectral data).
% Aims
% Method
Here we train this system using stars with known stellar mass labels
obtained from a training set of stars with both
\kepler\ asteroseismological observations (hence the known masses) and
\apogee\ infrared spectral data.
We find that (after training) \thecannon\ can infer stellar
masses---and therefore also stellar ages---for red-giant stars using
infrared spectral data alone.
% Results
We demonstrate the validity of the stellar masses and ages by three
methods:
First, we use cross-validation to demonstrate label accuracy on the
training set; we find mass accuracies of roughly XXXX and age accuracies
of roughly ZZZZ for typical-quality \apogee\ spectra.
Second, we look at the dependence on mass (or age) of
\thecannon\ spectral expectation, and show that it makes good physical
sense:
The age indicators in the spectral space are associated with elements
that can be ``dredged up'' or elements where chromospheric activity is
visible, or XXXX or YYYY.
Third, we use \thecannon\ to estimate age labels for the entire
\apogee\ sample, and show that the ages of stellar structures in the
Milky Way follow expectations, even conditioned on abundances.
All three of these tests show that we can obtain red-giant mass and
age information from stellar spectra; these capabilities open up new
opportunities for Milky Way and stellar astrophysics.
\end{abstract}

\keywords{%
keywords: incomplete!
---
methods: data analysis
---
methods: statistical
---
stars: abundances
---
stars: fundamental parameters
---
surveys
---
techniques: spectroscopic
}

\section{Introduction}\label{sec:Intro}

Asteroseismology surveys, such as \most, \corot, and \kepler, have
been extremely successful and productive in bringing us information
about stellar interiors and stellar parameters and ages.
These missions operate by taking high-cadence, high-precision stellar
photometry, in which stellar oscillation modes are visible in the
Fourier domain.
These missions have operated by taking long stretches of
uninterrupted, uniform, dense imaging data on single stars.
They are expensive missions, but absolutely critical to calibrate
physical stellar models and set standards for stellar parameter
estimation, all of which is required for the ultimate success of the
next generation of stellar surveys, particularly including the
\gaia\ Mission.
If \gaia\ is able to obtain stellar parameters that are both accurate
and precise, it will revolutionize our view of the Milky Way and its
formation.

In this \documentname\ we take a look at the combination of
asteroseismological and spectroscopic data for stellar parameter
estimation.
Our interest is in whether the power of asteroseismological
measurements made on a small subset of stars can be amplified to help
with parameter estimation on a much larger set, even those for which
no direct asterosiesmological measurements are possible.


\end{document}
